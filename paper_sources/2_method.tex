\section{Methodology}


\subsection{Segmentation Model Architecture}
In our analysis, we employed segmentation models based on encoder-decoder architectures for both patch-level and Whole Slide Image (WSI)-level segmentation, where the final segmentation was constructed via a sliding-window inference approach. 
We trained various combinations of encoders and decoders, ultimately selecting those models that demonstrated the best performance for each task based on rigorous cross-validation experiments.
Our analysis below refers to two segmentation models, both sharing the same image encoder, but with different decoders, which we will refer to as \textbf{\textit{model 1}} and \textbf{\textit{model 2}}.

For patch-level segmentation, the preferred architecture (\textbf{\textit{model 1}}) to experiment with as a Feature Pyramid Network (FPN) as the encoder, and a Mix Vision Transformer (MiT) as the decoder. 
The FPN is a popular choice for segmentation tasks due to its ability to extract multi-scale features efficiently \cite{lin2017feature}. 
It was initialized using pretrained weights from the ImageNet dataset to leverage prior knowledge and improve convergence during training \cite{qubvel2023segmentation}. 
FPN creates a top-down architecture with lateral connections to build high-level semantic feature maps at multiple scales. 
The decoder in this architecture was the Mix Vision Transformer (MiT), originally proposed in the SegFormer framework \cite{xie2021segformer}. 
MiT uses a transformer-based architecture to process and combine features extracted by the encoder. 
This transformer-based decoder proved effective in our experiments, especially in handling the features of patch-level inputs. 
The decoder takes the multi-scale features from the FPN and transforms them into a segmentation map, preserving both global and local context.

As a secondary model (\textbf{\textit{model 2}}), we trained a model with a different encoder, a U-Net++ initialized again from ImageNet-pretrained weights \cite{qubvel2023segmentation}. 
U-Net++ builds upon the classic U-Net architecture by adding nested, dense skip connections, which enhance the model's ability to learn fine-grained details and smooth transitions between different regions. The decoder for the WSI-level segmentation task was a ResNeSt101, chosen from the timm library \cite{qubvel2023segmentation}. 
ResNeSt (Split-Attention Networks) extends the traditional ResNet architecture by incorporating split-attention blocks, enabling the network to capture diverse features within a single layer.


\subsection{Training  Details}

For Task 1, we followed a similar training procedure to obtain the training hyper-parameters that worked the best and would optimize the cross-validation performance of \textbf{\textit{model 1}} and \textit{\textbf{model 2}}. The following details summarize the key aspects of the training process:

\begin{itemize}
\item \textbf{Optimizer:} We used the Nadam optimizer \cite{dozat2016}, which combines the benefits of the Nesterov accelerated gradient and Adam optimizers, offering faster convergence and improved stability during training.
    
\item \textbf{Learning Rate:} The learning rate was set to \(1 \times 10^{-4}\), a value chosen after initial hyperparameter tuning to balance the speed of convergence with model performance.
    
\item \textbf{Batch Size:} A batch size of 8 was used to manage the memory requirements while ensuring efficient gradient computation.
    
\item \textbf{Image size:} The input images were resized to \(1024 \times 1024\) pixels for \textbf{\textit{model 1}} and \(512 \times 512\) pixels for \textbf{\textit{model 2}}.
    
\item \textbf{Number of Epochs:} \textbf{\textit{model 1}} was trained for 20 epochs, whereas \textbf{\textit{model 2}} trained for 60 epochs. 
These numbers were chosen to ensure sufficient training time and guarantee convergence while monitoring the validation loss and the dice value to avoid overfitting.
    
\item \textbf{Loss Function:} The Dice loss function was used to handle the class imbalance in the dataset and improve segmentation performance by focusing on overlap between predicted and true segmentations. 
This was added to the Cross-Entropy loss, as this combination is known to provide benefits when dealing with overfitting and miscalibration %\cite{galdran_optimal_2023}.
\end{itemize}

Other details available on our github repository \href{https://github.com/agaldran/kpis}{\url{github.com/agaldran/kpis}}.

\subsection{Inference: from Local to Global segmentations}
For Task 1, inference was implemented with Test-Time Augmentation (flipping horizontally or/and vertically each patch), and the final segmentation was found by averaging the prediction of the five cross-validation models. 
For task 2, the code made available as a template by the organizers was employed. 
This is, an entire WSI is broken down into smaller patches of size $2048\times2048$, with a stride of $1024$ pixels in each direction, and then downsampled to the same resolution at which each model was trained. After this all patches are forwarded through a model to obtain the corresponding local segmentations, which are upsampled back to a $2048\times2048$ resolution.
These patches are then stitched together into a single WSI segmentation.

